\markboth{Introduction}{Introduction}
\section{Introduction}
In recent years, energy storage has been playing a more and more important role in various sectors. With the rapid advances in lithium-ion technology, the costs of lithium-ion batteries have been decreasing and their usage increasing steadily. Today, they are applied in many fields, e.g. Electric vehicles, storage in combination with photovoltaic systems, grid stabilization, etc. The various applications result in different operation strategies and stress factors, which often need to be modelled. The modelling of batteries makes it possible to simulate the behaviour of existing technologies while designing a system (i.e. with the purpose of determining the ideal type and size of the battery). It also plays an important role in the further development of the technologies themselves (i.e. of battery management systems). \\
Most battery models can be classified into 3 categories: (i) Theoretical, (ii) semi-empirical and (iii) empirical models~\cite{cui_multi-stress_2015, xu_degradation-limiting_2013}. The most precise approach is theoretical modelling, in which it is attempted to simulate the physical processes within the battery. Such processes could be the transfer of lithium ions through the electrolyte or the ageing mechanisms, e.g. lithium plating. A negative aspect of theoretical modelling is the complexity, which results in a high resource demand and thus slow simulations. A positive aspect is the fact that theoretical models can be adapted to any type of battery. The resource demand is significantly reduced with empirical models, in which certain behaviour is represented by equivalent circuits that are parametrised from measurements. However, simplifications must be made to a certain degree and an empirical model of one battery cannot be easily transferred to another, because the measurements have to be recorded for every model. In the worst-case scenario, the measured data might differ so strongly that the model has to be changed completely. The model used in this package takes a semi-empirical approach, in which the pros of the theoretical and empirical approaches are combined. The only con is the necessity of simplifications. Theoretical phenomena are used to model measured data, which allows for easy adaptation of the model to different technologies with low resource consumption. \\
A motivation for this project is the scarcity of detailed, cell-resolved simulation models~\cite{cordoba-arenas_control-oriented_2015} and the almost complete lack of open-source models. Furthermore, most \matlab\ simulations appear to be designed in a procedural style. Object oriented (OOP) programming in \matlab\ still seems to be a rare phenomenon today, although advanced OOP capabilities were introduced as early as version R2008a~\cite{foti_inside_2008}. This may in part be due to the fact that Matlab classes are often thought of as slow (with long method overhead times). However, the just-in-time compiler\footnote{Matlab's code execution mechanism.} (JIT) was overhauled in \matlab\ R2016a, resulting in an OOP performance increase of up to 40~\% compared to the predecessor~\cite{_matlab_2016}. It can be assumed that OOP performance will only increase further with newer releases. The approach chosen for this open-source model combines the flexibility of OOP design patterns with Matlab's highly optimized double-precision general matrix-matrix multiplication (dgemm) and vectorization libraries.