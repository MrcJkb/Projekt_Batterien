\markboth{Summary and outlook}{Summary and outlook}
\section{Summary and outlook}
The goal of this project was the development of a flexible, open-source battery pack model that compensates for the dearth of cell-resolved models in the field of battery simulation and can be easily parametrized using data sheets. This was accomplished by means of combining Matlab's highly optimized vectorization and dgemm libraries with OOP design patterns.
A semi-empirical model was successfully developed based on the data sheets of lithium iron phosphate (LFP) cells. Fist, the charging behaviour was replicated using curve fits of measured discharge curves - the cell voltage as functions of the discharged capacity for various currents. These curve fits were combined and used to interpolate the data for currents in between the recorded curves, making it possible to retrieve the voltage for any given current and discharge capacity. The simulated behaviour was validated by removing a measured curve from the data set and comparing it to one that was interpolated using the remaining curve fits. The validated charging behaviour was implemented via the Strategy design pattern, making it possible to swap out the LFP curve fitting classes with user-defined curve fits of other technologies' charging behaviour. A simple event oriented age model was implemented using the Observer design pattern, enabling a loose coupling of the battery pack and age model. This in turn facilitates the implementation of the age model on a cell level and on a simplified pack level as well as the extension with more detailed models that take additional ageing factors into account. A mathematical cycle counting algorithm was implemented and validated by means of comparison with the popular rainflow counting algorithm. \\
The combination of cells into any conceivable topology with either active or passive balancing of strings was made possible with a variation of the Composite design pattern. In order to ensure high performance, method delegation was limited to the components' getters and setters. It was further optimized with caching. By doing so, the actual charge iteration could be implemented on the pack level, rather than having to be realized for each individual cell. A short simulation of a battery pack revealed that it is necessary to differentiate between charging and discharging curves for more realistic behaviour. The ability to do so was implemented in the package. Furthermore, it was shown that a simulation with strongly fluctuating currents results in an equally vigorously oscillating pack voltages. An extension that makes it possible to even out such fluctuations was added to the package.\\ 
Once again using the Observer design pattern, the means for the simulation of CCCV charging combined with a cell-resolved BMS was provided. By limiting the cell-level computation to the setting of a logical flag on the pack level and delegating the charge current calculation to the pack, the workload was minimized. In order to enable more lightweight simulations using this package, a simplified version of the model was devised. Finally, GUI tools and a facade (the \mcode{batteryPack} class) were devised in order to provide the user with a simple interface to the complex subsystems of the model.\\
All in all, the goals set for this project were exceeded. Not only was a model created that can be easily adapted for various technologies. Components of the package can also be used separately and implemented into other models. For a proper validation, it would be necessary to take measurements from a battery in field tests and compare it to a model. This was not possible within the scope of this project due to access and time limitations. However, the test simulations provide a sufficient validation for many use cases. With the fast development of new lithium ion technologies, flexible, easily adaptable models are becoming more and more important. The package developed in the scope of this project provides a good solution. By making the source code freely accessible on GitHub\footnote{Available at: \url{https://github.com/MrcJkb/lfpBattery.git}}, further optimization and validation is facilitated.
\clearpage

\markboth{BSD license}{BSD license}
\section*{BSD license}
\addcontentsline{toc}{section}{BSD license}
Copyright \textcopyright\ 2017 Marc Jakobi, Festus Anyangbe, Marc Schmidt\newline
All rights reserved.
\newline
\newline
Redistribution and use in source and binary forms, with or without modification, are permitted provided that the following conditions are met:
\newline\newline
1. Redistributions of source code must retain the above copyright notice, this list of conditions and the following disclaimer.
\newline\newline
2. Redistributions in binary form must reproduce the above copyright notice, this list of conditions and the following disclaimer in the documentation and/or other materials provided with the distribution.
\newline\newline
3. Neither the name of the copyright holder nor the names of its contributors may be used to endorse or promote products derived from this software without specific prior written permission.
\newline\newline
THIS SOFTWARE IS PROVIDED BY THE COPYRIGHT HOLDERS AND CONTRIBUTORS "AS IS" AND ANY EXPRESS OR IMPLIED WARRANTIES, INCLUDING, BUT NOT LIMITED TO, THE IMPLIED WARRANTIES OF MERCHANTABILITY AND FITNESS FOR A PARTICULAR PURPOSE ARE DISCLAIMED. IN NO EVENT SHALL THE COPYRIGHT HOLDER OR CONTRIBUTORS BE LIABLE FOR ANY DIRECT, INDIRECT, INCIDENTAL, SPECIAL, EXEMPLARY, OR CONSEQUENTIAL
DAMAGES (INCLUDING, BUT NOT LIMITED TO, PROCUREMENT OF SUBSTITUTE GOODS OR SERVICES; LOSS OF USE, DATA, OR PROFITS; OR BUSINESS INTERRUPTION) HOWEVER CAUSED AND ON ANY THEORY OF LIABILITY, WHETHER IN CONTRACT, STRICT LIABILITY, OR TORT (INCLUDING NEGLIGENCE OR OTHERWISE) ARISING IN ANY WAY OUT OF THE USE OF THIS SOFTWARE, EVEN IF ADVISED OF THE POSSIBILITY OF SUCH DAMAGE.